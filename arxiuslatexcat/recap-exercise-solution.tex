\documentclass[12pt]{article}

\usepackage{url}

\title{Deu secrets per fer una bona xerrada científica}
\author{El teu nom}

\begin{document}
\maketitle

\section{Introduction}

El text d'aquest exercici és una versió significativament breu i lleugerament modificada de l'excel·lent article del mateix nom de Mark Schoeberl i Brian Toon:
\url {http://www.cgd.ucar.edu/cms/agu/scientific_talk.html}

\section{Els secrets}

He compilat aquesta llista personal de ``secrets´´ desprès d'escoltar conferenciants eficaços i ineficaços. No pretenc que aquesta llista sigui completa, però estic segur que hi ha coses que em queden fora. Però la meva llista probablement cobreix aproximadament el 90\% del que heu de saber i fer.

\begin{enumerate}
\item Prepareu el vostre material amb cura i lògicament. Explica una història.

\item Practiqueu la vostra xerrada. No hi ha excusa per manca de preparació.

\item No poseu massa material. Els bons conferenciants presenten un o dos punts centrals i s'adhereixen a ells.

\item Eviteu les equacions. Es diu que per a cada equació de la vostra xerrada es reduirà a la meitat el nombre de persones que l'entendran. És a dir, si q és el nombre d'equacions de la vostra xerrada i si n és el nombre de persones que entenen la vostra xerrada, resulta que

\begin{equation}
n = \gamma \left( \frac{1}{2} \right)^q
\end{equation}
on$\gamma$ és una constant de proporcionalitat.

\item Limiteu els punts de conclusió. La gent no pot recordar més d'un parell de coses d'una xerrada, sobretot si escolta moltes xerrades en grans reunions.

\item Parleu amb el públic i no amb la pantalla. Un dels problemes més freqüents que veig és que el conferenciant parla a la pantalla del projector.

\item Eviteu fer sons que distreguin. Intenteu evitar ``Ummm´´ o ``Ahhh´´ entre frases.

\item  Poleix els teus gràfics. Aquí teniu una llista de consells per obtenir millors gràfics:

\begin{itemize}
\item Utilitzeu lletres grans.

\item Manteniu els gràfics senzills. No mostris gràfics que no necessitaràs.

\item  Utilitzeu el color.
\end{itemize}

\item Accepteu amb amabilitat les preguntes.

\item  Utilitzeu l'humor si és possible. Sempre em sorprèn  veure com fins i tot una broma realment coixa fa riure en una xerrada de ciències.


\end{enumerate}

\end{document}
