\documentclass[12pt]{article}

\usepackage{url}

\begin{document}
Deu secrets per fer una bona xerrada científica

By: <tu>

Introducció

El text d'aquest exercici és una versió traduïda, significativament abreujada i lleugerament modificada de l'excel·lent article homònim de Mark Schoeberl i Brian Toon:
\ url {http://www.cgd.ucar.edu/cms/agu/scientific_talk.html}

Els secrets

He compilat aquesta llista personal de "secrets" per escoltar conferenciants eficaços i ineficaços. No pretenc que aquesta llista sigui completa; estic segur que hi ha coses que he deixat de banda. Però, probablement, la meva llista cobreix aproximadament el 90% del que heu de saber i fer.

1) Prepareu el material amb cura i lògica. Explica una història.

2) Practiqueu la vostra xerrada. No hi ha excusa per manca de preparació.

3) No poseu massa material. Els bons conferenciants tenen un o dos punts centrals i exploten aquest material.

4) Eviteu les equacions. Es diu que per a cada equació de la vostra xerrada es reduirà a la meitat el nombre de persones que l'entendran. És a dir, si q és el nombre d'equacions de la vostra xerrada i n el nombre de persones que entenen la vostra xerrada, resulta que

n = gamma (1/2) a la potència de q

on gamma és una constant de proporcionalitat.

5) Limiteu el nombre de punts de conclusió. La gent no pot recordar més d'un parell de coses d'una xerrada, sobretot si escolta moltes xerrades en grans reunions.

6) Parleu amb el públic i no amb la pantalla. Un dels problemes més habituals que veig és queel conferenciant parla a la pantalla.

7) Eviteu fer sons distractius. Intenteu evitar "Ummm" o "Ahhh" entre frases.

8) Poliu els gràfics. Aquí teniu una llista de consells per obtenir millors gràfics:

* Utilitzeu lletres grans.

* Mantingueu els gràfics senzills. No mostris gràfics que no necessitaràs.

* Utilitzeu el color.

9) Sigueu amables amb les preguntes.

10) Utilitzeu l'humor si és possible. Sempre em sorprèn que fins i tot d'una broma coixa se'n riguin en una xerrada de ciència.


\end{document}
