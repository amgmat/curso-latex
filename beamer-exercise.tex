\documentclass{beamer}

\begin{document}

Las cuentas del Gran Capitán

Perico de los Palotes

Reino de Castilla y León

fecha

Orden del día

* Objetivos
* Antecedentes
* Descripción general
* Conclusiones


¡Fuera del orden del día!

* Picos, palas y azadones, cien millones de ducados
* Frailes, monjas y pobres, ciento cincuenta mil ducados
* Guantes perfumados, cien mil ducados
* Reponer y arreglar las campanas, ciento sesenta mil ducados
* Pequeñeces del rey, cien millones de ducados 



Objetivos clave y  factores de éxito

*Campaña de Nápoles
 	** Conquistar el reino de Nápoles
	** Ganar acceso al resto de Italia



* La victoria ha sido total
* Los  recursos concedidos han sido escasos


Descripción general

figura cuentas.png


Tabla.% tamaño de la fuente \scriptsize


\begin{verbatim} % verbatim para reproducir las filas de la tabla, 
no tiene que ir en la solución.

Concepto & Coste & Coste acumulado \\

Picos, \ldots & 100.000 K & 100.000 K\\

Frailes, \ldots & 150 K & 100.150 K \\

Guantes \ldots & 100 K & 100.250 K \\

Campanas \ldots & 160 K & 100.410 K \\

Pequeñeces \ldots  & 100.000 K & 200.410 K \\

\end{verbatim}



Conclusiones

% dos columnas de tres ítems cada una 
* Hemos ganado la guerra
* Sin apenas tropas
* Con un presupuesto reducido


* Los soldados han mostrado arrojo y generosidad
* Han regalado un reino a España
* No se merecen estas pequeñeces


\end{document}




